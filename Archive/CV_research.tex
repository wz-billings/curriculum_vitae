

%\href{https://www.wcu.edu/learn/departments-schools-colleges/cas/science-and-math/biology/biology-faculty-staff/heather-coan.aspx}{\ul{PI: Heather Coan, Western Carolina University.}} \hfill Spring 2016 -- Present \\
%\indent Contact: \href{mailto:hacoan@email.wcu.edu}{hacoan@email.wcu.edu}
%\setlist{nolistsep}
%\begin{itemize}[noitemsep]
%\item Examined the ability of keratin to modulate the autophagy pathway in human cells.
%\item Maintained HEK-293 cultures for the lab group, and prepared cultures for experiments.
%\item Optimized transfection into HEK cells of a dual-reporter autophagy biomarker plasmid.
%\item Visualized changes in the autophagy pathway using wide-field epifluorescence microscopy.
%\item Compiled and analyzed data using FIJI and \texttt{R}.
%\end{itemize}
%
%\href{https://www.wcu.edu/learn/departments-schools-colleges/cas/science-and-math/mathcsdept/mathematics-and-computer-science-faculty-and-staff/andrew-penland.aspx}{\ul{PI: Andrew Penland, Western Carolina University.}} \hfill Fall 2019 -- Present \\
%\indent Contact: \href{mailto:adpenland@email.wcu.edu}{adpenland@email.wcu.edu}
%\setlist{nolistsep}
%\begin{itemize}[noitemsep]
%\item Simulated results from an Armitage-Doll carcinogenesis model on a large scale. 
%\item Graphically displayed results from large simulations.
%\item Instructed research group members in using \texttt{R} and the tidyverse suite.
%\item Helped solve problems related to probability distributions on hypergraphs.
%\end{itemize}
%
%\href{https://www.wcu.edu/learn/departments-schools-colleges/cas/science-and-math/mathcsdept/mathematics-and-computer-science-faculty-and-staff/jeff-lawson.aspx}{\ul{PI: Jeff Lawson, Western Carolina University.}} \hfill Fall 2019 -- Present \\
%\indent Contact: \href{mailto:jlawson@email.wcu.edu}{jlawson@email.wcu.edu}
%\setlist{nolistsep}
%\begin{itemize}[noitemsep]
%\item Analyzed least squares models for logistic growth ODE parameter recovery from time series data.
%\item Developed a suite of functions for replicable time series data generation and least squares fitting in \texttt{R}.
%\end{itemize}
%
%\href{https://publichealth.uga.edu/faculty-member/andreas-handel/}{\ul{PI: Andreas Handel, University of Georgia.}} \hfill Summer 2019 \\
%\indent Contact: \href{mailto:ahandel@uga.edu}{ahandel@uga.edu} \\
%\indent Website: \href{https://www.andreashandel.com/}{https://www.andreashandel.com/}
%\setlist{nolistsep}
%\begin{itemize}[noitemsep]
%\item Participant in the 2019 REU site Population Biology of Infectious Diseases at the University of Georgia, and was funded by the NIH.
%\item Cleaned and joined a complex relational data set from a \href{https://www.jabfm.org/content/32/2/226}{\ul{2016--2017 study}} at the University of Georgia student health center.
%\item Analyzed the impact of viral load estimates from qPCR data on clinical prognoses and treatment of influenza A. 
%\item Modeled and visualized the relationships between several variables using \texttt{R}, and presented results using \texttt{R}-Markdown.
%\end{itemize}
%
%\href{https://biology.uncg.edu/people/louis-marie-bobay-2/}{\ul{PI: Louis-Marie Bobay, UNC Greensboro.}} \hfill Summer 2018 \\
%\indent Contact: \href{mailto:ljbobay@uncg.edu}{ljbobay@uncg.edu} 
%\setlist{nolistsep}
%\begin{itemize}[noitemsep]
%\item Participant in the 2018 REU site Math-Biology at the University of North Carolina at Greensboro, and was funded by the NSF.
%\item Constructed a probabilistic model for determining the probability of $n$ number of convergent mutations between two bacterial genomes.
%\item Verified the accuracy of the model using Monte-Carlo simulation methods in \texttt{Python}.
%\item Visualized model predictions, effects of parameters, and simulation results in \texttt{R}.
%\item I also worked extensively with Dr. Johnathan Rowell (contact: \href{mailto:jtrowell@uncg.edu}{jtrowell@uncg.edu}) on the same project.
%\end{itemize}

\end{itemize}