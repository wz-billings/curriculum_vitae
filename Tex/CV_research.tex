\textbf{Research:}

\href{https://www.wcu.edu/learn/departments-schools-colleges/cas/science-and-math/biology/biology-faculty-staff/heather-coan.aspx}{\ul{PI: Heather Coan, Western Carolina University.}} \hfill Spring 2016 - Present
\setlist{nolistsep}
\begin{itemize}[noitemsep]
\item Examined the ability of keratin to modulate the autophagy pathway in human cells.
\item Maintained HEK-293 cultures for the lab group, and prepared cultures for experiments.
\item Optimized transfection into HEK cells of a dual-reporter autophagy biomarker plasmid.
\item Visualized changes in the autophagy pathway using wide-field epifluorescence microscopy.
\item Compiled and analyzed data using FIJI and \texttt{R}.
\end{itemize}

\href{https://publichealth.uga.edu/faculty-member/andreas-handel/}{\ul{PI: Andreas Handel, University of Georgia.}} \hfill Summer 2019
\setlist{nolistsep}
\begin{itemize}[noitemsep]
\item Participant in the 2019 REU site Population Biology of Infectious Diseases at the University of Georgia, and was funded by the NIH.
\item Cleaned and joined a complex relational data set from a \href{https://www.jabfm.org/content/32/2/226}{\ul{2016--2017 study}} at the University of Georgia student health center.
\item Analyzed the impact of viral load estimates from qPCR data on clinical prognoses and treatment of influenza A. 
\item Modeled and visualized the relationships between several variables using \texttt{R}, and presented results using \texttt{R}-Markdown.
\end{itemize}

\href{https://biology.uncg.edu/people/louis-marie-bobay-2/}{\ul{PI: Louis-Marie Bobay, UNC Greensboro.}} \hfill Summer 2018
\setlist{nolistsep}
\begin{itemize}[noitemsep]
\item Participant in the 2018 REU site Math-Biology at the University of North Carolina at Greensboro, and was funded by the NSF.
\item Constructed a probabilistic model for determining the probability of $n$ number of convergent mutations between two bacterial genomes.
\item Verified the accuracy of the model using Monte-Carlo simulation methods in \texttt{Python}.
\item Visualized model predictions, effects of parameters, and simulation results in \texttt{R}.
\end{itemize}

\href{https://www.wcu.edu/learn/departments-schools-colleges/cas/science-and-math/chemphys/faculty-and-staff/william-r.-kwochka.aspx}{\ul{PI: William Kwochka, Western Carolina University}} \hfill Fall 2017 - Spring 2018
\setlist{nolistsep}
\begin{itemize}[noitemsep]
\item Synthesized complex organic products, namely phenylboronic acid derivatives and Lewis acid-base complexes.
\item Analyzed the properties of these complexes for rotation activity, in the hopes that we would discover an effective molecular rotor.
\item Confirmed identity of samples using H-NMR, C-NMR, H-H homonuclear COSY, IR, and GC/MS methods.
\end{itemize}

\href{https://www.wcu.edu/learn/departments-schools-colleges/cas/science-and-math/biology/biology-faculty-staff/jeremy-hyman.aspx}{\ul{PI: Jeremy Hyman, Western Carolina University.}} \hfill Summer 2016
\setlist{nolistsep}
\begin{itemize}[noitemsep]
\item Participant in WCU's Summer Undergraduate Research Program in 2016 for incoming freshman, and was funded by WCU.
\item Observed aggressive behaviors of urban vs. rural song sparrows ({\it Melospiza melodia}).
\item Performed song playback experiments to quantify aggressive song sparrow behavior in the field.
\end{itemize}

\ul{OEIS Contributions}
\setlist{nolistsep}
\begin{itemize}[noitemsep]
\item Author of sequence \href{https://oeis.org/A319302}{\ul{A319302}}.
\item Wrote Python code and generated a b-file for sequence \href{https://oeis.org/A110529}{\ul{A110529}.}
\end{itemize}